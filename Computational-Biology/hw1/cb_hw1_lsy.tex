\documentclass{article}
\usepackage{hyperref}
\usepackage{parskip}
\setlength{\parskip}{1em}
\setlength{\parindent}{0em}

\title{Computational Biology Homework 1}
\author{Siyun Liu}
\date{}

\begin{document}

\maketitle

\section{Problem 1}

Website: \href{https://www.uniprot.org/uniprotkb/O95800/entry}{O95800 GPR75 Human}

Function: G protein-coupled receptor that is activated by the chemokine CCL5/RANTES. Probably coupled to heterotrimeric Gq proteins, it stimulates inositol trisphosphate production and calcium mobilization upon activation. Together with CCL5/RANTES, may play a role in neuron survival through activation of a downstream signaling pathway involving the PI3, Akt and MAP kinases. CCL5/RANTES may also regulate insulin secretion by pancreatic islet cells through activation of this receptor.

Publications: It can be checked in \href{https://www.uniprot.org/uniprotkb/O95800/publications}{this link}. Here I only list a few of them.

\begin{itemize}
    \item Tarttelin, E E et al. “Cloning and characterization of a novel orphan G-protein-coupled receptor localized to human chromosome 2p16.” Biochemical and biophysical research communications vol. 260,1 (1999): 174-80. doi:10.1006/bbrc.1999.0753
    \item Hillier, Ladeana W et al. “Generation and annotation of the DNA sequences of human chromosomes 2 and 4.” Nature vol. 434,7034 (2005): 724-31. doi:10.1038/nature03466
    \item Ota, Toshio et al. “Complete sequencing and characterization of 21,243 full-1em human cDNAs.” Nature genetics vol. 36,1 (2004): 40-5. doi:10.1038/ng1285
\end{itemize}

Pfam domain: PF00001

50\% similar proteins


\section{Problem 2}

Nucleic acid sequences: \href{https://www.ncbi.nlm.nih.gov/genbank/}{GenBank}, \href{https://www.ebi.ac.uk/ena/browser/home}{ENA}, \href{https://www.ddbj.nig.ac.jp/index-e.html}{DDBJ}

Protein sequences: \href{https://www.uniprot.org/}{UniProt}, \href{https://www.ncbi.nlm.nih.gov/protein/}{NCBI Protein}

Protein families: \href{http://pfam.xfam.org/}{Pfam}, \href{https://www.ebi.ac.uk/interpro/}{InterPro}

Protein structures: \href{https://www.rcsb.org/}{RCSB PDB}, \href{https://www.ebi.ac.uk/pdbe/scop/}{SCOP}, PDBe, PDBj, BMRB

\section{Problem 3}

Three major methods are NMR, X-ray and Cryo-EM.

Limits:

\begin{itemize}
    \item NMR: The size of the protein is limited to about 40 kDa, and the resolution is not as high as X-ray crystallography.
    \item X-ray: The protein must be crystallized, and the protein must be stable in the crystalline state.
    \item Cryo-EM: The resolution is not as high as X-ray crystallography (though it is making continuous progress), and the protein must be stable in the vitrified state.
\end{itemize}

\section{Problem 4}



\end{document}
