\documentclass{article}

\usepackage[UTF8]{ctex}
\usepackage{geometry}
\usepackage{graphicx}
\usepackage{subcaption}
\usepackage{float}

\usepackage[version=4]{mhchem}

\geometry{a4paper, top=2.54cm, bottom=2.54cm, left=3.18cm, right=3.18cm}
\setlength{\parindent}{0em}

\title{免疫组化实验报告}
\author{刘思昀 2022522011}
\date{2025年2月27日}

\begin{document}
\maketitle

\section{实验目的}

掌握免疫组织化学的原理和操作.

\section{实验原理}

免疫组织化学又称免疫细胞化学, 是指带显色剂标记的特异性抗体在组织细胞原位通过抗原抗体反应和组织化学的呈色反应, 对相应抗原进行定性、定位、定量测定的一项新技术。它把免疫反应的特异性、组织化学的可见性巧妙地结合起来, 借助显微镜的显像和放大作用, 在细胞、亚细胞水平检测各种抗原物质.

\section{实验材料与仪器}

\subsection{试剂}

\begin{itemize}
    \item $90\%$酒精 ($80\ ml$处理10张切片)
    \item $80\%$酒精 ($80\ ml$处理10张切片)
    \item $70\%$酒精 ($80\ ml$处理10张切片)
    \item 柠檬酸钠溶液: 1包柠檬酸钠粉末加水至$1000\ ml$溶解 ($1000\ ml$处理10张切片)
    \item 5x TBS ($50\ mM$): 5包1x TBS粉剂加水至$500\ ml$ ($500\ ml$处理10张切片)
    \item 1x TBS ($10\ mM$): 5x TBS溶液稀释得到,需要$6\ ml$
    \item $5\%$ BSA/TBS封闭液: $0.3\ g$ BSA + $6\ ml$ ($10\ mM$, 1x) TBS溶液 ($500\ \mu l$处理10张切片)
    \item 一抗: 将一抗溶液用$5\%$ BSA/TBS封闭液稀释至1:1000
    \item 发色溶液 (现用现配): $10\ ml$ DAB chromogen + $0.5\ ml$ DAB substrate配置得到
\end{itemize}

\subsection{仪器}

染缸, 摇床, 微波炉, 湿盒, 载片盒, 移液器, 冰箱等.

\section{实验步骤}

\subsection{脱蜡}

在装满二甲苯的染缸中依次进行以下操作: 浸泡$30\ min$, 上下洗20次, 上下洗20次, 浸泡$10\ min$中.

在装满不同浓度乙醇的染缸中上下洗20次, 染缸中乙醇的浓度依次为: $100\%$, $100\%$, $90\%$, $80\%$, $70\%$.

然后放在水槽中用流水洗$5\ min$, 注意不要使水流直接冲在样本上.

\subsection{第一次blocking (阻断内源性过氧化物酶活性)}

洗缸室温下置于摇床, 用超纯水配置的$3\% \ \ce{H2O2}$处理$30\ min$; 结束后流水洗$4\ min$. 

\subsection{抗原修复}

将柠檬酸钠缓冲液倒入保鲜盒, 微波炉最大火力 ($98^{\circ} C-100^{\circ} C$) 加热至沸腾. 

将切片放入金属篮中, 放入煮沸的柠檬酸钠溶液中, 煮沸$10-20\ min$; 期间注意补充液体防止煮干, 注意不要脱片, 一旦脱片停止加热.

冷却至室温后取出, 5x TBS洗涤3次, 每次$5\ min$.

\subsection{第二次blocking}

配置$5\%$ BSA/TBS溶液, 100-200ul/玻片 (注意:溶液一定要覆盖住所有组织) , 放在湿盒 (注意:湿盒中加少量水) 中, 室温放置40min, 甩去多余液体, 不洗; 

\subsection{一抗}

稀释后的一抗, 每张玻片滴加$100-200\ ul$, 完全盖住样本即可

放置于加少量水的湿盒中$4^{\circ} C$过夜.

\subsection{二抗}

用5x TBS冲洗一次, 放置装有5x TBS的洗缸中$20\ min$.

加二抗Complement溶液一滴 ($100\ \mu l$左右), 在加水的湿盒中室温放置$10\ min$, 然后用5x TBS冲洗两次.

加HRP conjugate溶液一滴 ($100\ \mu l$左右) , 在加水的湿盒中室温放置$15\ min$.

\subsection{制片}

用5x TBS清洗4次; 配制的DAB溶液, 每张玻片滴加$100-200\ \mu l$左右, 放置$5-10\ min$; 再用5x TBS冲洗4次.

用苏木精染核$1.5\ min$, 流水洗$5\ min$, 5x TBS中放置$3-5\ min$.

依次放入$70\%$乙醇1次$3\ min$, $95\%$乙醇1次$3\ min$, $100\%$乙醇2次, 各$3\ min$, 二甲苯$3\ min$, 最后滴加中性树胶封片.

\section{实验结果与分析}

阴性对照组中,见\ref{fig:sub1}, 细胞核呈现蓝色, 细胞质呈现淡蓝色, 细胞内无棕色, 说明阴性对照组中无靶标抗原存在.

实验组中,见\ref{fig:sub2}, 细胞核呈现蓝色, 细胞质呈现棕色, 说明实验组中有靶标抗原存在, 为阳性结果.

\begin{figure}[htbp]
    \centering
    \begin{subfigure}[b]{0.45\textwidth}
        \includegraphics[width=\textwidth]{negative.png}
        \caption{对照组, 物镜放大倍数: 40x}
        \label{fig:sub1}
    \end{subfigure}
    \hfill
    \begin{subfigure}[b]{0.45\textwidth}
        \includegraphics[width=\textwidth]{positive.png}
        \caption{实验组, 物镜放大倍数: 40x}
        \label{fig:sub2}
    \end{subfigure}
    \caption{染色后显微镜下观察家结果}
    \label{fig:group}
\end{figure}

\section{思考题}

查阅文献, 选取一篇你感兴趣的关于免疫组化在科研中应用的文章进行简单介绍, 包括免疫组化技术在该文章中如何应用以及其应用价值.

\vspace{0.5cm}

选取文章: Evaluation of methylene blue restaining versus conventional hydrogen peroxide decolorization in immunohistochemical diagnosis of melanoma.

DOI: https://doi.org/10.1038/s41598-025-89186-8

\vspace{0.5cm}

文章内容介绍和免疫组化技术的应用: 黑色素瘤的病理诊断依赖免疫组化技术检测标记物, 包括Melan A、HMB-45、PRAME、Ki-67等. 然而, 黑色素颗粒与IHC中DAB显色产物均为棕黄色, 导致显微镜下难以区分, 严重影响诊断准确性. 传统脱色方法利用过氧化氢虽然能够去除DAB显色产物, 但会导致组织损伤, 且脱色不完全. 文章中提出了一种新方法, 用美蓝染色后再用过氧化氢脱色, 使黑色素颗粒呈现蓝色, 使其与DAB显色产物的颜色差异增大, 从而提高了黑色素瘤的诊断准确性.

\vspace{0.5cm}

应用价值: 通过增大阳性与阴性样本之间的颜色差异, 使病理诊断时更容易区分黑色素与靶标染色,减少误判, 从而为黑色素瘤的早期诊断和精准病理评估提供可靠工具,改善患者预后管理。

\section{讨论}

\begin{enumerate}
    \item 苏木素复染时间需要摸索, 尤其要考虑阳性染色的位置.
    \item 切片脱蜡和水化要充分; 加反应液时要覆盖组织充分; 每次加液前甩干洗涤液, 但又防止
    干片.
    \item 一抗的清洗: 单独冲洗, 防止交叉反应造成污染; 温柔冲洗, 防止切片的脱落. 冲洗的时间要足够, 才能彻底洗去结合的物质.
    \item 苏木精染色$30\ s$比较合适, 避免颜色过深.
    \item 将样本放入柠檬酸钠加热时, 要时刻观察样本的状态, 可以缩短每次加热的时间, 多次加热, 方便及时补充挥发的液体, 还能避免脱片.
    \item 显微镜观察样本时, 注意不要对焦到玻片上, 而是对焦到样本层.
\end{enumerate}

\end{document}